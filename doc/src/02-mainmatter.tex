\chapter{Theory}
test \citep{orazem2008}


\chapter{Getting Starter}
\section{Introduction}\label{introduction}

\texttt{ecx} is a Fortran library providing formulas for
electrochemistry. C API allows usage from C, or can be used as a basis
for other wrappers. Python wrapper allows easy usage from Python.

To use \texttt{ecx} within your
\href{https://github.com/fortran-lang/fpm}{fpm} project, add the
following to your \texttt{fpm.toml} file:

\begin{verbatim}
    [dependencies]
    ecx = { git="https://github.com/MilanSkocic/ecx.git" }
\end{verbatim}

\section{Dependencies}\label{dependencies}

\begin{verbatim}
gcc>=10
gfortran>=10
fpm>=0.7
stdlib>=0.7
\end{verbatim}

\section{Installation}\label{installation}

A Makefile is provided, which uses
\href{https://fpm.fortran-lang.org}{fpm}, for building the library.

\begin{itemize}
\tightlist
\item
  On windows, \href{https://www.msys2.org}{msys2} needs to be installed.
  Add the msys2 binary (usually
  C:\textbackslash msys64\textbackslash usr\textbackslash bin) to the
  path in order to be able to use make.
\item
  On Darwin, the \href{https://formulae.brew.sh/formula/gcc}{gcc}
  toolchain needs to be installed.
\end{itemize}

Build: the configuration file will set all the environment variables
necessary for the compilation

\begin{verbatim}
    chmod +x configure.sh
    . ./configure.sh
    make
\end{verbatim}

Run tests

\begin{verbatim}
    make test
\end{verbatim}

Install

\begin{verbatim}
    make install
\end{verbatim}

Uninstall

\begin{verbatim}
    make uninstall
\end{verbatim}

\section{License}\label{license}

MIT






\chapter{API}
Functions, subroutines and parameters, that
are exposed in the API, are documented through man pages.
The latter are available for use on \emph{unix}-like systems.

\section{Fortran}\label{sec_api_fortran}
\input{build/latex/\NAME.tex}

\section{C}
The C API strictly defines the same the functions, subroutines and constants
defined in the Fortran API (see \$\ref{sec_api_fortran}). 
A prefix \code{\NAME \_} is added due to the lack of 
namespaces in the C language.

The arguments have the same names as those defined in the Fortran API and
have the same meaning. 
Additional arguments of type \code{size\_t} are necessary when passing pointers. 
The additional arguments providing the size of pointers (arrays) is 
named as following \code{n<POINTER\_ARGUMENT\_NAME>}. 
\fvset{frame=single}
\VerbatimInput{../include/\NAME .h}

\section{Python}


\appendix
\chapter{Changelog}
\section{Version 0.1.0-dev}\label{version-0.1.0-dev}

\begin{itemize}
\tightlist
\item
  Implementation of eis + C API
\item
  Python wrappers for eis.
\end{itemize}

