\section{Introduction}
    The \code{ecx} library is a collection of routines for electrochemistry.
    A C API allows usage from C, or can be used as a basis for other wrappers.
    A Python wrapper allows easy usage from Python.

    To use \code{ecx} within your
    \href{https://github.com/fortran-lang/fpm}{fpm} project, add the
    following to your \code{fpm.toml} file:

    \begin{verbatim}
        [dependencies]
        ecx = { git="https://github.com/MilanSkocic/ecx.git" }
    \end{verbatim}


\section{Dependencies}
    \begin{verbatim}
        gcc>=10
        gfortran>=10
        fpm>=0.7
        stdlib>=0.7
    \end{verbatim}

\section{Installation}


    A Makefile is provided, which uses
    \href{https://fpm.fortran-lang.org}{fpm}, for building the library.

    \begin{itemize}
    \item
      On windows, \href{https://www.msys2.org}{msys2} needs to be installed.
      Add the msys2 binary (usually
      C:\textbackslash msys64\textbackslash usr\textbackslash bin) to the
      path in order to be able to use make.
    \item
      On Darwin, the \href{https://formulae.brew.sh/formula/gcc}{gcc}
      toolchain needs to be installed.
    \end{itemize}

    \begin{verbatim}
        chmod +x configure.sh
        ./configure.sh
        make
        make test
        make install
        make uninstall
    \end{verbatim}
    
    You need a compiler that can compile the 
    \href{https://github.com/fortran-lang/stdlib}{stdlib}.


\section{License}

MIT
